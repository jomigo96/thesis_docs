% -- Extra symbols
\usepackage{amssymb}
\usepackage{amsthm}
\usepackage{empheq}
\DeclarePairedDelimiter{\ceil}{\lceil}{\rceil}
\DeclarePairedDelimiter{\floor}{\lfloor}{\rfloor}
\DeclarePairedDelimiter{\braces}{\{}{\}}
\DeclareMathOperator{\lcm}{\text{\slshape\mdseries\sffamily lcm}}
% override some commands so they don't get rm fonts
\let\max\relax
\DeclareMathOperator*{\max}{\text{\slshape\mdseries\sffamily max}}
\let\min\relax
\DeclareMathOperator*{\min}{\text{\slshape\mdseries\sffamily\text{min}}}
\let\mod\relax
\DeclareMathOperator{\mod}{\text{\slshape\mdseries\sffamily\text{mod}}}
\let\gcd\relax
\DeclareMathOperator{\gcd}{\text{\slshape\mdseries\sffamily\text{gcd}}}
\let\exp\relax
\DeclareMathOperator{\exp}{\text{\slshape\mdseries\sffamily\text{exp}}}
% user defined commands
\newrobustcmd*\bigO{\mathcal{O}}
\newrobustcmd*\mathdash{\text{-}}

% -- Bibliography and citations
\usepackage[
	backend = biber,
	style = numeric,
	sorting = ynt,
    maxbibnames = 8,
    maxcitenames = 2,
	%alldates=iso
	]{biblatex}
%\usepackage{fvextra}
\usepackage[autostyle]{csquotes}

% -- Glossary and nomenclature (usage with bib2gls)
\usepackage[record, symbols]{glossaries-extra}

% --  Floats/images
\graphicspath{{graphics/}}
\usepackage{caption}
\usepackage{subcaption}
\usepackage{flafter}
\usepackage{tabularx}
\usepackage{multirow}
\usepackage{ragged2e}
\usepackage{multicol}

% -- Drawing schemes and plotting
\usepackage{tikz}
\usepackage{pgfplots}
\usetikzlibrary{shapes, shapes.misc, shapes.symbols, positioning, decorations.pathreplacing, external, arrows.meta, patterns, plotmarks, hobby}
\tikzset{>=Latex} % solid arrow tip shape
\pgfplotsset{compat=1.16}
\pgfplotsset{table/search path = {data}}
\usepgfplotslibrary{statistics}
\pgfkeys{/pgf/number format/.cd,1000 sep={}} % remove comma as thousands separator

% -- Pseudocode and source code
\usepackage{algorithm}
\usepackage{algpseudocode}
\usepackage{verbatim}

\renewcommand{\textproc}{\texttt} % typeset algorithmic function names
% enable linking to algorithm line numbers. credit to https://tex.stackexchange.com/questions/148977/incorrect-reference-to-a-line-in-algorithmic-using-hyperref
\newcommand\plabel[1]{\phantomsection\label{#1}}
% define the for each loop in pseudocode. credit to https://tex.stackexchange.com/questions/149162/how-can-i-define-a-foreach-loop-on-the-basis-of-the-existing-forall-loop
\algnewcommand\algorithmicforeach{\textbf{for each}}
\algdef{S}[FOR]{ForEach}[1]{\algorithmicforeach\ #1\ \algorithmicdo}

% -- Tables
\usepackage{booktabs}
\usepackage{makecell}

% -- Misc
\usepackage[disable]{todonotes}
\usepackage{subfiles}

% -- Extra math - units
\usepackage[binary-units=true]{siunitx}

% --- Specific to the DIMA-scheduler thesis document ---

% -- custom colors
\colorlet{colorA}{blue!50!white}
\colorlet{colorB}{green!50!white}
\colorlet{colorC}{magenta!80!white}
\colorlet{colorD}{orange!90}
\colorlet{swred}{red!75!black!70!white}
\colorlet{hwgray}{gray!10!white}

% -- styles
\tikzstyle module_style_1=[hwgray, dashed, fill=cyan, fill opacity=0.1]
\tikzstyle module_style_2=[hwgray, dashed, fill=magenta, fill opacity=0.1]
\tikzstyle{decision} = [diamond, draw, text badly centered, node distance=3cm, inner sep=0pt, text width=1.8cm]
\tikzstyle{block} = [rectangle, draw, text centered, rounded corners, minimum height=1cm, inner sep=0pt, text width=3cm] 
\tikzstyle{line} = [draw,->]
\tikzstyle{file} = [tape, tape bend top=none, tape bend height=2mm, draw, minimum height=1cm, text width=2.6cm, inner sep=0pt, text centered]

% -- scripts
\directlua{require("scripts/tableformat.lua")}
\directlua{require("scripts/loadlines.lua")}
