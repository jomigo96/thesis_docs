%% arara: lualatex: {shell: yes, options: ["-aux-directory=build"]}
%% arara: biber: {options: ["-input-directory=build", "-output-directory=build"]}
%% arara: bib2gls: {group: yes, options: ["--dir=build", "--tex-encoding=utf-8"]}
%% arara: lualatex: {shell: yes, options: ["-aux-directory=build"]} 
% arara: lualatex: {shell: yes, options: ["-aux-directory=build"]} 

\documentclass[main.tex]{subfiles}

\begin{document}

\chapter{Optimization algorithms}
\label{an:algs}

\section{\glsfmtlong{sa} algorithm}
\label{an:sa}

\begin{algorithm}[H]
    \begin{algorithmic}[1]
        \directlua{load_lines("data/sa-alg.tex", 2, 24)}
        \algstore{myalg}
    \end{algorithmic}
    \caption{\gls{sa} algorithm for the partition scheduling problem.}
    \label{alg:sa}
\end{algorithm}

\begin{algorithm}[H]
    \ContinuedFloat
    \begin{algorithmic}
        \algrestore{myalg}
        \directlua{load_lines("data/sa-alg.tex", 25, 37)}
    \end{algorithmic}
    \caption{\gls{sa} algorithm for the partition scheduling problem. (cont.)}
\end{algorithm}

\section{Tabu search algorithm}
\label{an:tabu}

\begin{algorithm}[H]
    \begin{algorithmic}[1]
        \directlua{load_lines("data/tabu-alg.tex", 2, 22)}
        \algstore{myalg}
    \end{algorithmic}
    \caption{Tabu search algorithm for the partition scheduling problem.}
    \label{alg:tabu}
\end{algorithm}

\begin{algorithm}[H]
    \ContinuedFloat
    \begin{algorithmic}
        \algrestore{myalg}
        \directlua{load_lines("data/tabu-alg.tex", 23, 38)}
    \end{algorithmic}
    \caption{Tabu search algorithm for the partition scheduling problem. (cont.)}
\end{algorithm}

\section{Genetic algorithm}
\label{an:genetic}

\begin{algorithm}[H]
    \begin{algorithmic}[1]
        \directlua{load_lines("data/genetic-alg.tex", 2, 18)}
        \algstore{myalg}
    \end{algorithmic}
    \caption{Genetic algorithm for the partition scheduling problem.}
    \label{alg:genetic}
\end{algorithm}

\begin{algorithm}[H]
    \ContinuedFloat
    \begin{algorithmic}
        \algrestore{myalg}
        \directlua{load_lines("data/genetic-alg.tex", 19, 40)}
    \end{algorithmic}
    \caption{Genetic algorithm for the partition scheduling problem. (cont.)}
\end{algorithm}

\chapter{Test Cases}
\label{an:test-cases}

% This is heavy on compilation, ignore for now
% Much of this section was auto-generated. Fix formating as needed.

\setstretch{1.0}

\section{\texorpdfstring{$2M6P$}{2M6P}}

\begin{table}[H]
    \centering
    \caption{Problem specification for $2M6P$.}
    \begin{minipage}[b]{0.55\linewidth}
        \centering
        \begin{tabular}{ccccc}
            \toprule
            Partition   & Period       & Duration   & Memory    & Domain       \\
            \midrule
            1 & \num{1000} & \num{1} & \num{9} & All modules \\
2 & \num{1000} & \num{31} & \num{9} & All modules \\
3 & \num{500} & \num{5} & \num{5} & All modules \\
4 & \num{100} & \num{3} & \num{4} & All modules \\
5 & \num{100} & \num{10} & \num{1} & All modules \\
6 & \num{100} & \num{5} & \num{1} & All modules \\
            \bottomrule
        \end{tabular}
        \subcaption{Partition information.}
    \end{minipage}%
    \begin{minipage}[b]{0.41\linewidth}
        \centering
        \begin{tabular}{cc}
            \toprule
            Module   &   Memory   \\
            \midrule
            1 & \num{36} \\
2 & \num{36} \\
            \bottomrule
        \end{tabular}
        \subcaption{Module information.}
    \end{minipage}%
\end{table}


\section{\texorpdfstring{$4M10P$}{4M10P}}

\begin{table}[H]
    \centering
    \caption{Problem specification for $4M10P$.}
    \begin{minipage}[b]{0.55\linewidth}
        \centering
        \begin{tabular}{ccccc}
            \toprule
            Partition   & Period       & Duration   & Memory    & Domain     \\
            \midrule
            1 & \num{1000} & \num{23} & \num{8} & All modules \\
2 & \num{1000} & \num{41} & \num{7} & All modules \\
3 & \num{1000} & \num{56} & \num{8} & All modules \\
4 & \num{1000} & \num{77} & \num{6} & All modules \\
5 & \num{1000} & \num{35} & \num{8} & All modules \\
6 & \num{500} & \num{14} & \num{7} & All modules \\
7 & \num{500} & \num{14} & \num{5} & All modules \\
8 & \num{200} & \num{1} & \num{7} & All modules \\
9 & \num{200} & \num{12} & \num{3} & All modules \\
10 & \num{100} & \num{8} & \num{7} & All modules \\
            \bottomrule
        \end{tabular}
        \subcaption{Partition information.}
    \end{minipage}%
    \begin{minipage}[b]{0.41\linewidth}
        \centering
        \begin{tabular}{cc}
            \toprule
            Module   &   Memory   \\
            \midrule
            1 & \num{32} \\
2 & \num{31} \\
3 & \num{32} \\
4 & \num{31} \\
            \bottomrule
        \end{tabular}
        \subcaption{Module information.}

        \bigskip
        \begin{tabular}{lr}
            \toprule
            Chain   &   Delay   \\
            \midrule
            $p_{8} \rightarrow p_{7}$   &   \num{121} \\
$p_{3} \rightarrow p_{1}$   &   \num{842} \\
$p_{8} \rightarrow p_{6}$   &   \num{123} \\
            \bottomrule
        \end{tabular}
        \subcaption{Chain constraints.}
    \end{minipage}%

    \bigskip

    \begin{tabular}{|c|c|c|c|c|}
        \hline
        $m,n$ & 1 & 2 & 3 & 4 \\\hline
1 & - & \num{25} & \num{5} & \num{12}\\\hline
2 & \num{25} & - & \num{6} & \num{6}\\\hline
3 & \num{5} & \num{6} & - & \num{12}\\\hline
4 & \num{12} & \num{6} & \num{12} & -\\\hline
    \end{tabular}
    \subcaption{Network delays.}
\end{table}

Other constraints: $f_{8} ≠ f_{9}$, $f_{5} ≠ f_{10}$.

\section{\texorpdfstring{$4M20P$}{4M20P}}

\begin{table}[H]
    \centering
    \caption{Problem specification for $4M20P$.}
    \begin{minipage}[b]{0.55\linewidth}
        \centering
        \begin{tabular}{ccccc}
            \toprule
            Partition   & Period       & Duration   & Memory    & Domain       \\
            \midrule
            1 & \num{1000} & \num{46} & \num{6} & All modules \\
2 & \num{1000} & \num{29} & \num{8} & All modules \\
3 & \num{1000} & \num{73} & \num{1} & All modules \\
4 & \num{500} & \num{16} & \num{4} & All modules \\
5 & \num{500} & \num{29} & \num{9} & All modules \\
6 & \num{500} & \num{29} & \num{3} & All modules \\
7 & \num{500} & \num{5} & \num{6} & All modules \\
8 & \num{500} & \num{40} & \num{8} & All modules \\
9 & \num{500} & \num{24} & \num{1} & All modules \\
10 & \num{200} & \num{1} & \num{3} & All modules \\
11 & \num{200} & \num{21} & \num{9} & All modules \\
12 & \num{200} & \num{18} & \num{8} & All modules \\
13 & \num{200} & \num{21} & \num{2} & All modules \\
14 & \num{200} & \num{15} & \num{7} & All modules \\
15 & \num{200} & \num{23} & \num{2} & All modules \\
16 & \num{200} & \num{3} & \num{7} & All modules \\
17 & \num{100} & \num{9} & \num{6} & All modules \\
18 & \num{100} & \num{1} & \num{6} & All modules \\
19 & \num{100} & \num{2} & \num{5} & All modules \\
20 & \num{100} & \num{11} & \num{8} & All modules \\
            \bottomrule
        \end{tabular}
        \subcaption{Partition information.}
    \end{minipage}%
    \begin{minipage}[b]{0.41\linewidth}
        \centering
        \begin{tabular}{cc}
            \toprule
            Module   &   Memory   \\
            \midrule
            1 & \num{60} \\
2 & \num{63} \\
3 & \num{62} \\
4 & \num{63} \\
            \bottomrule
        \end{tabular}
        \subcaption{Module information.}

        \bigskip
        \begin{tabular}{lr}
            \toprule
            Chain   &   Delay   \\
            \midrule
            $p_{7} \rightarrow p_{14}$   &   \num{98} \\
$p_{13} \rightarrow p_{9}$   &   \num{125} \\
$p_{18} \rightarrow p_{5}$   &   \num{459} \\
$p_{9} \rightarrow p_{2}$   &   \num{550} \\
$p_{10} \rightarrow p_{11}$   &   \num{134} \\
$p_{19} \rightarrow p_{3}$   &   \num{582} \\
$p_{15} \rightarrow p_{1}$   &   \num{1003} \\
$p_{5} \rightarrow p_{19}$   &   \num{52} \\
            \bottomrule
        \end{tabular}
        \subcaption{Chain constraints.}
    \end{minipage}%

    \bigskip

    \begin{tabular}{|c|c|c|c|c|}
        \hline
        $m,n$ & 1 & 2 & 3 & 4 \\\hline
1 & - & \num{13} & \num{10} & \num{13}\\\hline
2 & \num{13} & - & \num{11} & \num{18}\\\hline
3 & \num{10} & \num{11} & - & \num{16}\\\hline
4 & \num{13} & \num{18} & \num{16} & -\\\hline
    \end{tabular}
    \subcaption{Network delays.}
\end{table}

Other constraints: $f_{6} ≠ f_{19}$, $f_{14} ≠ f_{4}$, $f_{12} ≠ f_{18}$, $f_{15} ≠ f_{14}$, $f_{7} ≠ f_{2}$, $f_{13} ≠ f_{11}$.

\section{\texorpdfstring{$8M40P$}{8M40P}}

\begin{table}[H]
    \centering
    \caption{Problem specification for $8M40P$.}
    \begin{minipage}[b]{0.55\linewidth}
        \centering
        \begin{tabular}{ccccc}
            \toprule
            Partition   & Period       & Duration   & Memory    & Domain       \\
            \midrule
            1 & \num{1000} & \num{1} & \num{1} & All modules \\
2 & \num{1000} & \num{64} & \num{4} & All modules \\
3 & \num{1000} & \num{80} & \num{1} & All modules \\
4 & \num{1000} & \num{42} & \num{6} & All modules \\
5 & \num{1000} & \num{37} & \num{3} & All modules \\
6 & \num{1000} & \num{49} & \num{8} & All modules \\
7 & \num{1000} & \num{16} & \num{2} & All modules \\
8 & \num{500} & \num{8} & \num{6} & All modules \\
9 & \num{500} & \num{23} & \num{1} & All modules \\
10 & \num{500} & \num{10} & \num{7} & All modules \\
11 & \num{500} & \num{37} & \num{7} & All modules \\
12 & \num{500} & \num{29} & \num{7} & All modules \\
13 & \num{500} & \num{35} & \num{6} & All modules \\
14 & \num{500} & \num{18} & \num{5} & All modules \\
15 & \num{500} & \num{34} & \num{8} & All modules \\
16 & \num{500} & \num{15} & \num{1} & All modules \\
17 & \num{500} & \num{11} & \num{2} & All modules \\
18 & \num{500} & \num{30} & \num{3} & All modules \\
19 & \num{500} & \num{22} & \num{3} & All modules \\
20 & \num{500} & \num{30} & \num{3} & All modules \\
21 & \num{500} & \num{31} & \num{1} & All modules \\
22 & \num{200} & \num{2} & \num{3} & All modules \\
23 & \num{200} & \num{9} & \num{5} & All modules \\
24 & \num{200} & \num{21} & \num{3} & All modules \\
25 & \num{200} & \num{21} & \num{5} & All modules \\
26 & \num{200} & \num{13} & \num{8} & All modules \\
27 & \num{200} & \num{19} & \num{9} & All modules \\
28 & \num{200} & \num{18} & \num{8} & All modules \\
29 & \num{200} & \num{12} & \num{3} & All modules \\
30 & \num{200} & \num{22} & \num{7} & All modules \\
31 & \num{200} & \num{8} & \num{9} & All modules \\
32 & \num{100} & \num{2} & \num{8} & All modules \\
33 & \num{100} & \num{5} & \num{8} & All modules \\
34 & \num{100} & \num{2} & \num{3} & All modules \\
35 & \num{100} & \num{14} & \num{1} & All modules \\
36 & \num{100} & \num{13} & \num{3} & All modules \\
37 & \num{100} & \num{9} & \num{8} & All modules \\
38 & \num{100} & \num{14} & \num{2} & All modules \\
39 & \num{100} & \num{10} & \num{2} & All modules \\
40 & \num{100} & \num{2} & \num{7} & All modules \\
            \bottomrule
        \end{tabular}
        \subcaption{Partition information.}
    \end{minipage}%
    \begin{minipage}[b]{0.41\linewidth}
        \centering
        \begin{tabular}{cc}
            \toprule
            Module   &   Memory   \\
            \midrule
            1 & \num{61} \\
2 & \num{64} \\
3 & \num{59} \\
4 & \num{64} \\
5 & \num{65} \\
6 & \num{60} \\
7 & \num{62} \\
8 & \num{60} \\
            \bottomrule
        \end{tabular}
        \subcaption{Module information.}

        \bigskip
        \begin{tabular}{lr}
            \toprule
            Chain   &   Delay   \\
            \midrule
            $p_{21} \rightarrow p_{26}$   &   \num{252} \\
$p_{23} \rightarrow p_{7}$   &   \num{215} \\
$p_{37} \rightarrow p_{18}$   &   \num{424} \\
$p_{1} \rightarrow p_{30}$   &   \num{166} \\
$p_{10} \rightarrow p_{25}$   &   \num{50} \\
$p_{27} \rightarrow p_{6}$   &   \num{706} \\
$p_{28} \rightarrow p_{34}$   &   \num{119} \\
$p_{30} \rightarrow p_{20}$   &   \num{259} \\
$p_{17} \rightarrow p_{14}$   &   \num{321} \\
$p_{21} \rightarrow p_{38}$   &   \num{109} \\
$p_{8} \rightarrow p_{18}$   &   \num{309} \\
$p_{5} \rightarrow p_{30}$   &   \num{204} \\
$p_{18} \rightarrow p_{23}$   &   \num{205} \\
$p_{8} \rightarrow p_{28}$   &   \num{157} \\
$p_{32} \rightarrow p_{24}$   &   \num{243} \\
            \bottomrule
        \end{tabular}
        \subcaption{Chain constraints.}
    \end{minipage}%

    \bigskip


\end{table}
\begin{table}[H]
	\ContinuedFloat
	\centering
    \begin{tabular}{|c|c|c|c|c|c|c|c|c|}
        \hline
        $m,n$ & 1 & 2 & 3 & 4 & 5 & 6 & 7 & 8 \\\hline
1 & - & \num{9} & \num{25} & \num{16} & \num{10} & \num{11} & \num{10} & \num{7}\\\hline
2 & \num{9} & - & \num{23} & \num{11} & \num{18} & \num{18} & \num{7} & \num{15}\\\hline
3 & \num{25} & \num{23} & - & \num{15} & \num{11} & \num{15} & \num{20} & \num{11}\\\hline
4 & \num{16} & \num{11} & \num{15} & - & \num{5} & \num{6} & \num{17} & \num{7}\\\hline
5 & \num{10} & \num{18} & \num{11} & \num{5} & - & \num{5} & \num{7} & \num{23}\\\hline
6 & \num{11} & \num{18} & \num{15} & \num{6} & \num{5} & - & \num{25} & \num{17}\\\hline
7 & \num{10} & \num{7} & \num{20} & \num{17} & \num{7} & \num{25} & - & \num{21}\\\hline
8 & \num{7} & \num{15} & \num{11} & \num{7} & \num{23} & \num{17} & \num{21} & -\\\hline
    \end{tabular}
    \subcaption{Network delays.}
\end{table}

Other constraints: $f_{31} ≠ f_{29}$, $f_{34} ≠ f_{9}$, $f_{15} ≠ f_{25}$, $f_{11} ≠ f_{40}$, $f_{38} ≠ f_{29}$, $f_{35} ≠ f_{10}$, $f_{32} ≠ f_{9}$, $f_{4} ≠ f_{25}$, $f_{12} ≠ f_{30}$, $f_{16} ≠ f_{25}$, $f_{28} = f_{29}$, $f_{2} = f_{8}$, $f_{4} = f_{15}$, $f_{31} = f_{38}$.

\section{\texorpdfstring{$20M100P$}{20M100P}}

\begin{table}[H]
    \centering
    \caption{Problem specification for $20M100P$.}
    \begin{minipage}[b]{0.55\linewidth}
        \centering
        \begin{tabular}{ccccc}
            \toprule
            Partition   & Period       & Duration   & Memory    & Domain       \\
            \midrule
            1 & \num{1000} & \num{55} & \num{2} & All modules \\
2 & \num{1000} & \num{78} & \num{1} & All modules \\
3 & \num{1000} & \num{28} & \num{4} & All modules \\
4 & \num{1000} & \num{2} & \num{6} & All modules \\
5 & \num{1000} & \num{75} & \num{6} & All modules \\
6 & \num{1000} & \num{53} & \num{7} & All modules \\
7 & \num{1000} & \num{47} & \num{7} & All modules \\
8 & \num{1000} & \num{32} & \num{2} & All modules \\
9 & \num{1000} & \num{46} & \num{1} & All modules \\
10 & \num{1000} & \num{42} & \num{1} & All modules \\
11 & \num{1000} & \num{68} & \num{6} & All modules \\
12 & \num{1000} & \num{48} & \num{7} & All modules \\
13 & \num{1000} & \num{79} & \num{2} & All modules \\
14 & \num{1000} & \num{45} & \num{8} & All modules \\
15 & \num{1000} & \num{74} & \num{9} & All modules \\
16 & \num{1000} & \num{12} & \num{7} & All modules \\
17 & \num{1000} & \num{46} & \num{7} & All modules \\
18 & \num{1000} & \num{80} & \num{5} & All modules \\
19 & \num{1000} & \num{24} & \num{6} & All modules \\
20 & \num{1000} & \num{38} & \num{1} & All modules \\
21 & \num{1000} & \num{45} & \num{8} & All modules \\
22 & \num{1000} & \num{78} & \num{9} & All modules \\
23 & \num{1000} & \num{21} & \num{8} & All modules \\
24 & \num{500} & \num{24} & \num{1} & All modules \\
25 & \num{500} & \num{23} & \num{9} & All modules \\
26 & \num{500} & \num{37} & \num{2} & All modules \\
27 & \num{500} & \num{15} & \num{4} & All modules \\
28 & \num{500} & \num{40} & \num{4} & All modules \\
29 & \num{500} & \num{40} & \num{8} & All modules \\
30 & \num{500} & \num{11} & \num{1} & All modules \\
31 & \num{500} & \num{13} & \num{8} & All modules \\
32 & \num{500} & \num{7} & \num{2} & All modules \\
33 & \num{500} & \num{17} & \num{7} & All modules \\
34 & \num{500} & \num{9} & \num{3} & All modules \\
35 & \num{500} & \num{24} & \num{6} & All modules \\
36 & \num{500} & \num{40} & \num{5} & All modules \\
37 & \num{500} & \num{4} & \num{6} & All modules \\
38 & \num{500} & \num{24} & \num{4} & All modules \\
39 & \num{500} & \num{34} & \num{7} & All modules \\
40 & \num{500} & \num{11} & \num{6} & All modules \\
		\bottomrule
        \end{tabular}
        \subcaption{Partition information. (1/3)}
    \end{minipage}%
    \begin{minipage}[b]{0.41\linewidth}
        \centering
        \begin{tabular}{cc}
            \toprule
            Module   &   Memory   \\
            \midrule
            1 & \num{62} \\
2 & \num{63} \\
3 & \num{60} \\
4 & \num{59} \\
5 & \num{65} \\
6 & \num{60} \\
7 & \num{63} \\
8 & \num{65} \\
9 & \num{62} \\
10 & \num{62} \\
11 & \num{62} \\
12 & \num{65} \\
13 & \num{63} \\
14 & \num{61} \\
15 & \num{62} \\
16 & \num{63} \\
17 & \num{60} \\
18 & \num{65} \\
19 & \num{59} \\
20 & \num{59} \\
            \bottomrule
        \end{tabular}
        \subcaption{Module information.}
    \end{minipage}%
\end{table}

\begin{table}[H]
	\ContinuedFloat
	\centering
	\begin{minipage}[b]{0.55\linewidth}
        \centering
        \begin{tabular}{ccccc}
            \toprule
            Partition   & Period       & Duration   & Memory    & Domain     \\
            \midrule
			41 & \num{500} & \num{5} & \num{4} & All modules \\
			42 & \num{500} & \num{31} & \num{1} & All modules \\
			43 & \num{500} & \num{16} & \num{1} & All modules \\
			44 & \num{500} & \num{17} & \num{2} & All modules \\
			45 & \num{500} & \num{6} & \num{3} & All modules \\
            46 & \num{500} & \num{30} & \num{2} & All modules \\
			47 & \num{500} & \num{23} & \num{6} & All modules \\
			48 & \num{500} & \num{3} & \num{6} & All modules \\
			49 & \num{200} & \num{11} & \num{4} & All modules \\
			50 & \num{200} & \num{9} & \num{4} & All modules \\
			51 & \num{200} & \num{19} & \num{2} & All modules \\
			52 & \num{200} & \num{12} & \num{2} & All modules \\
			53 & \num{200} & \num{16} & \num{7} & All modules \\
			54 & \num{200} & \num{15} & \num{2} & All modules \\
			55 & \num{200} & \num{15} & \num{7} & All modules \\
			56 & \num{200} & \num{13} & \num{4} & All modules \\
			57 & \num{200} & \num{16} & \num{9} & All modules \\
			58 & \num{200} & \num{21} & \num{1} & All modules \\
			59 & \num{200} & \num{17} & \num{2} & All modules \\
			60 & \num{200} & \num{9} & \num{2} & All modules \\
			61 & \num{200} & \num{18} & \num{6} & All modules \\
			62 & \num{200} & \num{17} & \num{9} & All modules \\
			63 & \num{200} & \num{17} & \num{6} & All modules \\
			64 & \num{200} & \num{12} & \num{3} & All modules \\
			65 & \num{200} & \num{16} & \num{5} & All modules \\
			66 & \num{200} & \num{12} & \num{9} & All modules \\
			67 & \num{200} & \num{23} & \num{9} & All modules \\
			68 & \num{200} & \num{3} & \num{6} & All modules \\
			69 & \num{200} & \num{4} & \num{2} & All modules \\
			70 & \num{200} & \num{23} & \num{9} & All modules \\
			71 & \num{200} & \num{20} & \num{4} & All modules \\
			72 & \num{200} & \num{18} & \num{4} & All modules \\
			73 & \num{100} & \num{8} & \num{5} & All modules \\
			74 & \num{100} & \num{4} & \num{4} & All modules \\
			75 & \num{100} & \num{12} & \num{4} & All modules \\
			76 & \num{100} & \num{8} & \num{2} & All modules \\
			77 & \num{100} & \num{1} & \num{4} & All modules \\
			78 & \num{100} & \num{1} & \num{6} & All modules \\
			79 & \num{100} & \num{3} & \num{6} & All modules \\
			80 & \num{100} & \num{8} & \num{2} & All modules \\
			81 & \num{100} & \num{15} & \num{3} & All modules \\
			82 & \num{100} & \num{1} & \num{9} & All modules \\
			83 & \num{100} & \num{3} & \num{4} & All modules \\
			84 & \num{100} & \num{10} & \num{9} & All modules \\
			85 & \num{100} & \num{10} & \num{2} & All modules \\
			86 & \num{100} & \num{2} & \num{2} & All modules \\
			87 & \num{100} & \num{5} & \num{5} & All modules \\
			88 & \num{100} & \num{14} & \num{4} & All modules \\
			89 & \num{100} & \num{10} & \num{8} & All modules \\
			90 & \num{100} & \num{7} & \num{3} & All modules \\
			\bottomrule
        \end{tabular}
        \subcaption{Partition information. (2/3)}
    \end{minipage}%
    \begin{minipage}[b]{0.41\linewidth}
        \centering
        \begin{tabular}{lr}
            \toprule
            Chain   &   Delay   \\
            \midrule
            $p_{85} \rightarrow p_{86}$   &   \num{120} \\
$p_{39} \rightarrow p_{54}$   &   \num{143} \\
$p_{41} \rightarrow p_{1}$   &   \num{684} \\
$p_{12} \rightarrow p_{50}$   &   \num{217} \\
$p_{78} \rightarrow p_{100}$   &   \num{118} \\
$p_{70} \rightarrow p_{68}$   &   \num{56} \\
$p_{83} \rightarrow p_{24}$   &   \num{111} \\
$p_{94} \rightarrow p_{79}$   &   \num{96} \\
$p_{6} \rightarrow p_{60}$   &   \num{141} \\
$p_{36} \rightarrow p_{56}$   &   \num{87} \\
$p_{3} \rightarrow p_{89}$   &   \num{77} \\
$p_{57} \rightarrow p_{67}$   &   \num{130} \\
$p_{95} \rightarrow p_{43}$   &   \num{324} \\
$p_{9} \rightarrow p_{64}$   &   \num{242} \\
$p_{73} \rightarrow p_{36}$   &   \num{481} \\
$p_{8} \rightarrow p_{33}$   &   \num{285} \\
$p_{21} \rightarrow p_{43}$   &   \num{264} \\
$p_{75} \rightarrow p_{25}$   &   \num{384} \\
$p_{29} \rightarrow p_{33}$   &   \num{423} \\
$p_{5} \rightarrow p_{30}$   &   \num{260} \\
$p_{53} \rightarrow p_{8}$   &   \num{788} \\
$p_{5} \rightarrow p_{44}$   &   \num{215} \\
$p_{4} \rightarrow p_{73}$   &   \num{52} \\
$p_{19} \rightarrow p_{96}$   &   \num{100} \\
$p_{38} \rightarrow p_{13}$   &   \num{640} \\
$p_{46} \rightarrow p_{19}$   &   \num{528} \\
$p_{67} \rightarrow p_{70}$   &   \num{180} \\
$p_{82} \rightarrow p_{33}$   &   \num{460} \\
$p_{56} \rightarrow p_{74}$   &   \num{84} \\
$p_{82} \rightarrow p_{99}$   &   \num{98} \\
$p_{1} \rightarrow p_{42}$   &   \num{425} \\
$p_{69} \rightarrow p_{99}$   &   \num{118} \\
$p_{60} \rightarrow p_{33}$   &   \num{500} \\
$p_{99} \rightarrow p_{86}$   &   \num{61} \\
$p_{30} \rightarrow p_{33}$   &   \num{165} \\
$p_{73} \rightarrow p_{70}$   &   \num{229} \\
$p_{6} \rightarrow p_{96}$   &   \num{77} \\
$p_{54} \rightarrow p_{1}$   &   \num{642} \\
$p_{68} \rightarrow p_{60}$   &   \num{138} \\
$p_{2} \rightarrow p_{21}$   &   \num{704} \\
            \bottomrule
        \end{tabular}
        \subcaption{Chain constraints.}
    \end{minipage}%
\end{table}

\begin{table}[H]
	\ContinuedFloat
	\centering
        \begin{tabular}{ccccc}
            \toprule
            Partition   & Period       & Duration   & Memory    & Domain     \\
            \midrule
			91 & \num{100} & \num{15} & \num{5} & All modules \\
			92 & \num{100} & \num{1} & \num{6} & All modules \\
			93 & \num{100} & \num{13} & \num{4} & All modules \\
			94 & \num{100} & \num{7} & \num{5} & All modules \\
			95 & \num{100} & \num{14} & \num{1} & All modules \\
			96 & \num{100} & \num{7} & \num{9} & All modules \\
			97 & \num{100} & \num{3} & \num{8} & All modules \\
			98 & \num{100} & \num{1} & \num{2} & All modules \\
			99 & \num{100} & \num{6} & \num{6} & All modules \\
			100 & \num{100} & \num{2} & \num{3} & All modules \\
			\bottomrule
        \end{tabular}
        \subcaption{Partition information. (3/3)}

		\bigskip
		{\scriptsize
		\begin{tabular}{|c|c|c|c|c|c|c|c|c|c|c|c|c|c|c|c|c|c|c|c|c|}
        \hline
        $m,n$ & 1 & 2 & 3 & 4 & 5 & 6 & 7 & 8 & 9 & 10 & 11 & 12 & 13 & 14 & 15 & 16 & 17 & 18 & 19 & 20 \\\hline
1 & - & \num{17} & \num{11} & \num{12} & \num{12} & \num{8} & \num{25} & \num{16} & \num{19} & \num{11} & \num{21} & \num{17} & \num{23} & \num{12} & \num{5} & \num{17} & \num{12} & \num{18} & \num{6} & \num{8}\\\hline
2 & \num{17} & - & \num{6} & \num{11} & \num{25} & \num{18} & \num{13} & \num{24} & \num{11} & \num{18} & \num{24} & \num{6} & \num{22} & \num{11} & \num{6} & \num{7} & \num{23} & \num{11} & \num{5} & \num{17}\\\hline
3 & \num{11} & \num{6} & - & \num{20} & \num{15} & \num{5} & \num{5} & \num{8} & \num{22} & \num{17} & \num{24} & \num{20} & \num{5} & \num{17} & \num{25} & \num{16} & \num{6} & \num{13} & \num{13} & \num{20}\\\hline
4 & \num{12} & \num{11} & \num{20} & - & \num{14} & \num{16} & \num{24} & \num{6} & \num{21} & \num{22} & \num{20} & \num{7} & \num{16} & \num{13} & \num{11} & \num{19} & \num{11} & \num{23} & \num{18} & \num{25}\\\hline
5 & \num{12} & \num{25} & \num{15} & \num{14} & - & \num{20} & \num{23} & \num{6} & \num{11} & \num{20} & \num{14} & \num{21} & \num{9} & \num{7} & \num{25} & \num{15} & \num{6} & \num{24} & \num{20} & \num{15}\\\hline
6 & \num{8} & \num{18} & \num{5} & \num{16} & \num{20} & - & \num{15} & \num{24} & \num{8} & \num{13} & \num{8} & \num{17} & \num{6} & \num{6} & \num{21} & \num{24} & \num{6} & \num{9} & \num{17} & \num{25}\\\hline
7 & \num{25} & \num{13} & \num{5} & \num{24} & \num{23} & \num{15} & - & \num{16} & \num{5} & \num{9} & \num{12} & \num{23} & \num{23} & \num{18} & \num{9} & \num{13} & \num{9} & \num{19} & \num{20} & \num{12}\\\hline
8 & \num{16} & \num{24} & \num{8} & \num{6} & \num{6} & \num{24} & \num{16} & - & \num{20} & \num{23} & \num{18} & \num{25} & \num{22} & \num{21} & \num{10} & \num{7} & \num{7} & \num{10} & \num{10} & \num{13}\\\hline
9 & \num{19} & \num{11} & \num{22} & \num{21} & \num{11} & \num{8} & \num{5} & \num{20} & - & \num{5} & \num{19} & \num{15} & \num{5} & \num{18} & \num{23} & \num{25} & \num{8} & \num{15} & \num{14} & \num{9}\\\hline
10 & \num{11} & \num{18} & \num{17} & \num{22} & \num{20} & \num{13} & \num{9} & \num{23} & \num{5} & - & \num{21} & \num{5} & \num{21} & \num{20} & \num{19} & \num{20} & \num{12} & \num{22} & \num{19} & \num{21}\\\hline
11 & \num{21} & \num{24} & \num{24} & \num{20} & \num{14} & \num{8} & \num{12} & \num{18} & \num{19} & \num{21} & - & \num{14} & \num{24} & \num{11} & \num{16} & \num{19} & \num{19} & \num{18} & \num{21} & \num{19}\\\hline
12 & \num{17} & \num{6} & \num{20} & \num{7} & \num{21} & \num{17} & \num{23} & \num{25} & \num{15} & \num{5} & \num{14} & - & \num{17} & \num{17} & \num{19} & \num{18} & \num{18} & \num{13} & \num{23} & \num{23}\\\hline
13 & \num{23} & \num{22} & \num{5} & \num{16} & \num{9} & \num{6} & \num{23} & \num{22} & \num{5} & \num{21} & \num{24} & \num{17} & - & \num{21} & \num{14} & \num{15} & \num{9} & \num{22} & \num{20} & \num{22}\\\hline
14 & \num{12} & \num{11} & \num{17} & \num{13} & \num{7} & \num{6} & \num{18} & \num{21} & \num{18} & \num{20} & \num{11} & \num{17} & \num{21} & - & \num{8} & \num{19} & \num{16} & \num{13} & \num{24} & \num{6}\\\hline
15 & \num{5} & \num{6} & \num{25} & \num{11} & \num{25} & \num{21} & \num{9} & \num{10} & \num{23} & \num{19} & \num{16} & \num{19} & \num{14} & \num{8} & - & \num{5} & \num{7} & \num{9} & \num{23} & \num{15}\\\hline
16 & \num{17} & \num{7} & \num{16} & \num{19} & \num{15} & \num{24} & \num{13} & \num{7} & \num{25} & \num{20} & \num{19} & \num{18} & \num{15} & \num{19} & \num{5} & - & \num{7} & \num{10} & \num{15} & \num{25}\\\hline
17 & \num{12} & \num{23} & \num{6} & \num{11} & \num{6} & \num{6} & \num{9} & \num{7} & \num{8} & \num{12} & \num{19} & \num{18} & \num{9} & \num{16} & \num{7} & \num{7} & - & \num{12} & \num{12} & \num{15}\\\hline
18 & \num{18} & \num{11} & \num{13} & \num{23} & \num{24} & \num{9} & \num{19} & \num{10} & \num{15} & \num{22} & \num{18} & \num{13} & \num{22} & \num{13} & \num{9} & \num{10} & \num{12} & - & \num{9} & \num{16}\\\hline
19 & \num{6} & \num{5} & \num{13} & \num{18} & \num{20} & \num{17} & \num{20} & \num{10} & \num{14} & \num{19} & \num{21} & \num{23} & \num{20} & \num{24} & \num{23} & \num{15} & \num{12} & \num{9} & - & \num{9}\\\hline
20 & \num{8} & \num{17} & \num{20} & \num{25} & \num{15} & \num{25} & \num{12} & \num{13} & \num{9} & \num{21} & \num{19} & \num{23} & \num{22} & \num{6} & \num{15} & \num{25} & \num{15} & \num{16} & \num{9} & -\\\hline
    \end{tabular}}
    \subcaption{Network delays.}
\end{table}

Other constraints: $f_{10} ≠ f_{5}$, $f_{39} ≠ f_{53}$, $f_{49} ≠ f_{40}$, $f_{74} ≠ f_{4}$, $f_{92} ≠ f_{31}$, $f_{40} ≠ f_{12}$, $f_{46} ≠ f_{19}$, $f_{35} ≠ f_{23}$, $f_{95} ≠ f_{6}$, $f_{32} ≠ f_{45}$, $f_{59} ≠ f_{76}$, $f_{51} ≠ f_{37}$, $f_{87} ≠ f_{9}$, $f_{41} ≠ f_{61}$, $f_{58} ≠ f_{8}$, $f_{24} ≠ f_{56}$, $f_{22} ≠ f_{11}$, $f_{9} ≠ f_{25}$, $f_{17} ≠ f_{68}$, $f_{34} = f_{82}$, $f_{65} = f_{77}$, $f_{86} = f_{90}$, $f_{29} = f_{97}$, $f_{34} = f_{54}$, $f_{54} = f_{82}$, $f_{33} = f_{96}$.


\section{\texorpdfstring{$3M15P\mathdash S$}{3M15PS}}

\begin{table}[H]
    \centering
    \caption{Problem specification for $3M15P\mathdash S$.}
        \centering
        \begin{tabular}{ccccccc}
            \toprule
            Partition   & Period       & Duration   & Memory    & Domain & Preemption points & Deadline      \\
            \midrule
1 & \num{1000} & \num{100} & \num{5} & All modules & $\braces*{\num{40},\num{80}}$ & \num{200}\\
2 & \num{1000} & \num{50} & \num{3} & All modules  & $\braces*{}$ & -- \\
3 & \num{1000} & \num{100} & \num{3} & All modules & $\braces*{\num{25},\num{40},\num{50},\num{60},\num{75}}$ & \num{300} \\
4 & \num{1000} & \num{70} & \num{3} & All modules  & $\braces*{\num{20},\num{40}}$ & \num{220}\\
5 & \num{500} & \num{35} & \num{4} & All modules   & $\braces*{}$ & -- \\
6 & \num{500} & \num{25} & \num{4} & All modules   & $\braces*{}$ & -- \\
7 & \num{500} & \num{50} & \num{2} & All modules   & $\braces*{}$ & -- \\
8 & \num{250} & \num{50} & \num{5} & All modules   & $\braces*{\num{20},\num{25},\num{30},\num{35}}$ & \num{175}\\
9 & \num{250} & \num{20} & \num{6} & All modules   & $\braces*{}$ & -- \\
10 & \num{200} & \num{30} & \num{7} & All modules  & $\braces*{}$ & -- \\
11 & \num{100} & \num{10} & \num{4} & All modules  & $\braces*{}$ & -- \\
12 & \num{100} & \num{20} & \num{2} & All modules  & $\braces*{}$ & -- \\
13 & \num{100} & \num{5} & \num{5} & All modules   & $\braces*{}$ & -- \\
14 & \num{100} & \num{10} & \num{5} & All modules  & $\braces*{}$ & -- \\
15 & \num{50} & \num{2} & \num{6} & All modules    & $\braces*{}$ & -- \\
            \bottomrule
        \end{tabular}
        \subcaption{Partition information.}

        \bigskip
    \begin{minipage}[b]{0.33\linewidth}
        \centering
        \begin{tabular}{cc}
            \toprule
            Module   &   Memory   \\
            \midrule
            1 & \num{33} \\
            2 & \num{35} \\
            3 & \num{30} \\
            \bottomrule
        \end{tabular}
        \subcaption{Module information.}
    \end{minipage}%
    \begin{minipage}[b]{0.33\linewidth}
        \centering
        \begin{tabular}{lr}
            \toprule
            Link   &   Delay   \\
            \midrule
            $p_{6} \rightarrow p_{15}$   &   \num{48} \\
$p_{10} \rightarrow p_{5}$   &   \num{125} \\
$p_{11} \rightarrow p_{12}$   &   \num{119} \\
            \bottomrule
        \end{tabular}
        \subcaption{Link constraints.}
    \end{minipage}%
    \begin{minipage}[b]{0.33\linewidth}
        \centering
        \begin{tabular}{|c|c|c|c|}
            \hline
            Modules & 1 & 2 & 3 \\\hline
            1 & - & \num{12} & \num{7}\\\hline
            2 & \num{12} & - & \num{14}\\\hline
            3 & \num{7} & \num{14} & -\\\hline
        \end{tabular}
    \subcaption{Network delays.}
    \end{minipage}%
\end{table}

Other constraints: $f_{8} ≠ f_{1}$, $f_{9} ≠ f_{5}$, $f_{10} ≠ f_{8}$, $f_{1} = f_{12}, ε=1$.

\setstretch{1.5}


\begin{comment}
\chapter{Tool overview}
\label{an:tool}

Workflow, features, limitations, I/O...

Problem specification via yaml files.

Processing options yields logs with messages detailing constraints that dont make sense. etc.
Run modes, in `analysis' mode, the tool is able to detail the evaluation function of the schedule, each specific partition's utility, which constraints are not met, etc.
In `solve' mode, the tool solves the problem according to other options.
The user can configure the number of steps, target values, algorithm to use, its parameters, etc.
Configuration is made via config file as well, or through via the command line interface.

Tool stashes results in serialized format, and can resume from previous states.
Outputs consist in table format, serialized format which can be (carefully) edited directly by the user, and in graphics similar to those in the body of this thesis.

Program flow chart...

\newpage
\begin{figure}[H]
    \centering
    \resizebox{!}{0.95\textheight}{\begin{tikzpicture}[scale=1.0, node distance=0pt]

    \node[block] (start) {start};
    \node[block, below = 16mm of start] (load) {load problem data};
    \node[file, above left = 0mm and 16mm of load] (file1) {\texttt{problem.yaml}};
    \node[file, below left = 0mm and 16mm of load] (file2) {Program configuration};
    \node[decision, below = 16mm of load] (select) {mode?};
    \node[file, right = 16mm of load] (insights) {constraint insights};
    \node[block, below left = 16mm and 20mm of select.south] (solve) {set exit condition $\alpha\geq 1$};
    \node[block, below = 16mm of select.south] (opt) {set exit condition \texttt{max\_steps}};
    \node[block, right = 55mm of select] (process) {load solution};
    \node[file, above = 16mm of process] (sol) {solution file (\texttt{.yaml})};
    \node[decision, below = 16mm of opt] (cache) {cached state?};
    \node[file, right = 16mm of cache, dashed] (state) {\texttt{state.yaml}};
    \node[block, below left = 16mm and 8mm of cache] (loop) {solution loop};
    \node[block, below right = 16mm and 8mm of cache] (csp) {initial assignment with CSP};
    \node[decision, below = 10mm of csp] (csp_success) {success?};
    \node[block, below = 16mm of csp_success] (exit) {exit (problem infeasible)};
    \node[decision, below=10mm of loop] (exit_cond) {exit condition?};
    \node[block, below=24mm of exit_cond] (analyse) {process solution};
    \node[block, below=24mm of analyse] (exit_success) {exit success};
    \node[file, below right = 12mm and 5mm of analyse] (out1) {solution/state (\texttt{.yaml})};
    \node[file, right = 4mm of out1] (out2) {formated solution (\texttt{.xml})};
    \node[file, right = 4mm of out2] (out3) {graphical solution (\texttt{.tex})};
    \node[file, right = 4mm of out3] (out4) {tabular schedule};

    %%%%%%%%%%%%%%% %%%%%%%%%% %%%%%%%%%%%%

    \path[line] (start) -- (load);
    \path[line] (file1.east) --++(6mm,0) |- ([yshift=2mm]load.west);
    \path[line] (file2.east) --++(6mm,0) |- ([yshift=-2mm]load.west);
    \path[line] (load) -- (select);
    \path[line] (load) -- (insights);
    \path[line] (select) -| (solve);
    \path[line] (select) -- node[fill=white] (l1) {`optimize'} (opt);
    \path[line] (select) -- node[fill=white] {`process'} (process);
    \node[fill=white] at (l1-|solve) {`solve'};
    \path[line] (sol) -- (process);
    \path[draw] (solve) |- ([yshift=3mm]cache.north);
    \path[line] (opt) -- (cache);
    \path[line] (state) -- (cache);
    \path[line] (cache) -- node[fill=white] {yes} (loop.north);
    \path[line] (cache) -- node[fill=white] {no} (csp.north);
    \path[line] (csp) -- (csp_success);
    \path[line] (csp_success) -- node[fill=white] {no} (exit);
    \path[line] (csp_success.west) --++(-20mm, 0) node[fill=white] {yes} |- (loop.east);
    \path[line] (loop) -- (exit_cond);
    \path[line] (exit_cond.west) --++(-8mm, 0) |- node[fill=white] {no} (loop.west);
    \path[line] (exit_cond) -- node[fill=white] {yes} (analyse);
    \path[line] (process) |- (analyse);
    \path[line] (analyse) -- (exit_success);
    \path[line, dashed] ([yshift=-5mm]analyse.south) -| (out4);
    \path[line, dashed, <-] (out3.north) --++(0,7mm);
    \path[line, dashed, <-] (out2.north) --++(0,7mm);
    \path[line, dashed, <-] (out1.north) --++(0,7mm);

\end{tikzpicture}
}
    \caption{Simplified scheduling tool flowchart.}
    \label{fig:flowchart}
\end{figure}

\end{comment}


\end{document}
